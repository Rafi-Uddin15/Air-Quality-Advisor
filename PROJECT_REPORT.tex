\documentclass[conference]{IEEEtran}
\IEEEoverridecommandlockouts
\usepackage{cite}
\usepackage{amsmath,amssymb,amsfonts}
\usepackage{algorithmic}
\usepackage{graphicx}
\usepackage{textcomp}
\usepackage{xcolor}
\usepackage{float}
\usepackage{hyperref}
\usepackage{booktabs}
\usepackage{longtable}

\def\BibTeX{{\rm B\kern-.05em{\sc i\kern-.025em b}\kern-.08em
    T\kern-.1667em\lower.7ex\hbox{E}\kern-.125emX}}

\begin{document}

\title{AI-Powered Indoor Environmental Advisor: An IoT and Machine Learning Approach}

\author{\IEEEauthorblockN{Rafi Uddin (2022-1-60-037), Md. Sakib Hasan (2022-1-60-098), \\
Sumona Sharmin Israt (2022-1-60-066), Md. Hasibul Hassan Himel (2022-3-60-113)}
\IEEEauthorblockA{\textit{Group: IOT-GRP-07} \\
\textit{Course: CSE406 -- Internet of Things}}
}

\maketitle

\begin{abstract}
This report presents the design and implementation of an intelligent air quality monitoring system that integrates low-cost IoT hardware with an advanced Machine Learning (ML) pipeline. Utilizing an Arduino Uno R3 controller, the system gathers real-time environmental data (Temperature, Humidity, Gas Levels) via DHT11 and MQ-135 sensors. A Random Forest Regressor model is employed to predict a standardized Air Quality Index (AQI) from raw sensor voltages, achieving a Mean Absolute Error (MAE) of 2.15. The system features a novel ``Glassmorphism'' dashboard built with Streamlit, which provides real-time health advice using a Retrieval-Augmented Generation (RAG) approach. This solution addresses the need for accessible, context-aware environmental monitoring in smart homes, bridging the gap between raw data and actionable health insights.
\end{abstract}

\begin{IEEEkeywords}
IoT, Machine Learning, Air Quality Monitoring, Arduino, Random Forest, Smart Home.
\end{IEEEkeywords}

\section{Introduction}
The quality of the air we breathe indoors is a fundamental determinant of public health, productivity, and overall well-being. With people spending approximately 90\% of their time indoors, the monitoring and management of Indoor Air Quality (IAQ) have become critical research frontiers. Poor IAQ is linked to a myriad of health issues, ranging from Sick Building Syndrome (SBS) to long-term respiratory and cardiovascular diseases. In the post-pandemic era, the demand for intelligent, real-time, and accessible environmental monitoring systems has surged, driven by a heightened awareness of airborne pathogens and pollutants.

Traditionally, air quality monitoring was the domain of expensive, industrial-grade equipment operated by professionals. However, the advent of the Internet of Things (IoT) has democratized access to environmental data. As noted by \cite{REHVA_AI_2025}, the convergence of low-cost sensing technologies and Artificial Intelligence (AI) is actively reshaping the landscape of building management systems. Current research emphasizes not just the collection of data, but the transformation of this data into actionable intelligence. This shift is crucial because raw pollutant concentrations (e.g., in ppm or $\mu g/m^3$) are often unintelligible to the average building occupant.

Recent advancements have focused on the deployment of sensor networks that can provide granular, room-level data. For instance, \cite{Montreal_Qatar_2025} demonstrated the efficacy of IoT sensor technology in a comparative case study between Montreal and Qatar, highlighting how localized monitoring can identify specific regional pollution patterns. Similarly, \cite{Dublin_Sensor_2025} explored the long-term viability of these low-cost sensor networks in Dublin, addressing the critical challenges of sensor drift and maintenance over extended periods.

Despite these advances, a significant gap remains in the interpretability of this data. While modern systems are proficient at data acquisition, they often fail to provide context-aware advice. A comprehensive review by \cite{Review_IoT_AI_2023} underscores the potential of AI-driven innovations to close this loop, moving from simple monitoring to active control and user guidance. However, many existing commercial solutions remain "black boxes," offering little insight into *why* air quality is degrading or *what* specific actions a user should take.

This project addresses these challenges by developing an ``Indoor Environmental Advisor.'' Unlike passive monitors, this system integrates an AI-driven advisory layer. By leveraging the foundational work seen in educational environments \cite{IoT_Edu_2025}, where intelligent systems manage student health, our proposed solution targets the smart home sector. It combines a robust hardware sensing layer with a Machine Learning model for accurate AQI prediction and a Generative AI interface to deliver personalized health recommendations.

The primary contributions of this report are:
\begin{itemize}
    \item The development of a cost-effective, Arduino-based multi-sensor node for real-time IAQ monitoring.
    \item The implementation of a Random Forest Regression model to correlate raw sensor voltages with standardized AQI metrics.
    \item The creation of an interactive, user-centric dashboard that utilizes RAG (Retrieval-Augmented Generation) to interpret environmental data for the layperson.
\end{itemize}

\section{Literature Review}
The field of Indoor Air Quality monitoring has witnessed explosive growth in literature between 2022 and 2025, primarily driven by the availability of affordable microcontrollers and the accessibility of powerful machine learning libraries. This section reviews recent methodologies, focusing on sensor integration, predictive modeling algorithms, and system architecture.

\subsection{IoT Sensor Technologies and Low-Cost Monitoring}
The cornerstone of modern IAQ systems is the low-cost sensor. \cite{LowCost_IoT_2024} details the design and prototyping of compact monitoring nodes that utilize electrochemical sensors like the MQ series. Their work highlights the trade-off between cost and precision, a theme echoed by \cite{Hashmy_Calib_2023}, who emphasize the absolute necessity of rigorous calibration protocols. Without advanced calibration, low-cost sensors are prone to significant cross-sensitivity errors, particularly in environments with varying temperature and humidity.

In the context of specific implementations, \cite{Ganesh_IoT_2023} proposed a smart IoT-based system that not only monitors parameters but employs predictive analytics to forecast trends. Their work demonstrates that even simple sensor arrays, when coupled with cloud connectivity, can provide utility comparable to much more expensive legacy systems.

\subsection{Machine Learning Algorithms for AQI Prediction}
The transition from data logging to predictive intelligence is defined by the choice of Machine Learning algorithm. Various approaches have been tested in the recent literature (2022-2024) to map non-linear sensor responses to accurate pollutant concentrations.

\cite{IoT_ML_Forecast_2022} provided a comparative analysis of KNN, ARIMA, and LSTM models. Their findings suggested that while time-series models like LSTM offer superior performance for long-term forecasting, lighter models like KNN or Random Forest are often sufficient for real-time edge computing applications. Building on this, \cite{Nurcahyanto_RNN_2022} implemented a Multilevel Recurrent Neural Network (RNN) specifically for PM10 prediction in industrial clean rooms, achieving high accuracy but at the cost of significant computational resources.

For more constrained environments, \cite{AQ_Ind_Std_2024} evaluated XGBoost, Random Forest, and SVM. They concluded that ensemble tree-based methods (like Random Forest) offer an optimal balance of interpretability and accuracy, handling the non-linear interference of temperature on gas sensor readings effectively. This aligns with the methodological choice in our project.

Deep learning approaches also continue to gain traction. \cite{IoT_ML_AQTR_2024} presented a system utilizing Neural Network Time Series for predicting dangerous gas concentrations. Their approach allows for proactive ventilation control, turning the monitoring system into an active safety device. Similarly, \cite{Gupta_AI_PM25_2023} investigated various AI techniques for AQI classification, highlighting the importance of feature engineering—specifically the inclusion of meteorological variables—in improving model robustness.

\subsection{System Integration and Applications}
Beyond the algorithms, the integration of these components into usable systems has been a key focus. \cite{Ind_AQ_Pred_2024} focused on the unique challenges of industrial environments, where higher pollutant concentrations require robust hardware housing and failsafe data transmission protocols.
\cite{LowCost_Pred_2025} recently demonstrated a fully integrated pipeline that leverages historical data for predictive modeling, allowing building managers to anticipate poor air quality events before they occur.

Table \ref{tab:lit_review} summarizes the key literature reviewed, highlighting the methodologies and primary outcomes of each study.

\begin{table*}[htbp]
\caption{Summary of Relevant Research Papers (2022-2025)}
\centering
\begin{tabular}{|p{1.5cm}|p{0.8cm}|p{4.5cm}|p{4.0cm}|p{6.5cm}|}
\hline
\textbf{Ref.} & \textbf{Year} & \textbf{Focus Area} & \textbf{Algorithm/Method} & \textbf{Key Outcome/Finding} \\
\hline
\cite{IoT_Edu_2025} & 2025 & Intelligent Education Environments & LSTM, Decision Tree & Developed a proactive management system for classrooms using complex event processing. \\
\hline
\cite{LowCost_Pred_2025} & 2025 & Predictive Modelling & Predictive Analytics & Demonstrated that low-cost sensors, when corrected with models, yield actionable data. \\
\hline
\cite{REHVA_AI_2025} & 2025 & AI in Management (Review) & Review of AI Techniques & AI is shifting from monitoring to autonomous control and optimization of HVAC systems. \\
\hline
\cite{IoT_ML_AQTR_2024} & 2024 & Safety \& Control & Neural Networks (Time Series) & High accuracy in predicting gas leaks and triggering automated ventilation. \\
\hline
\cite{AQ_Ind_Std_2024} & 2024 & Algorithm Comparison & XGBoost, RF, SVM & Ensemble methods (RF/XGBoost) generally outperform single classifiers for AQI tasks. \\
\hline
\cite{LowCost_IoT_2024} & 2024 & Low-cost Prototyping & Prototyping & Validated the feasibility of widespread, decentralized monitoring nodes. \\
\hline
\cite{Review_IoT_AI_2023} & 2023 & Innovation Review & Review & Identified the lack of user-centric interpretation in current commercial IoT devices. \\
\hline
\cite{Ganesh_IoT_2023} & 2023 & Smart IoT System & Predictive Analytics & Real-time trend identification improves user response time to pollution events. \\
\hline
\cite{Hashmy_Calib_2023} & 2023 & Sensor Calibration & Calibration Algorithms & Machine learning calibration significantly reduces sensor drift errors. \\
\hline
\cite{Gupta_AI_PM25_2023} & 2023 & PM2.5 Forecasting & Feature Engineering & Adding weather data (Temp/Hum) drastically improves pollutant prediction accuracy. \\
\hline
\cite{IoT_ML_Forecast_2022} & 2022 & Forecasting Systems & KNN, ARIMA, LSTM & LSTM is superior for temporal dependencies; ARIMA is viable for linear trends. \\
\hline
\cite{Nurcahyanto_RNN_2022} & 2022 & Industrial Clean Rooms & Multilevel RNN & Deep learning captures complex, non-linear pollutant dynamics in controlled spaces. \\
\hline
\end{tabular}
\end{center}
\label{tab:lit_review}
\end{table*}

\section{System Architecture}
The proposed system follows a modular architecture consisting of three main layers: Sensing, Processing, and Presentation. The design prioritizes modularity to ensure that individual components can be upgraded without overhauling the entire system.

\subsection{Hardware Design}
The hardware unit is built around the \textbf{Arduino Uno R3} (Fig. \ref{fig:uno}). It was selected over the ESP8266 NodeMCU due to its superior Analog-to-Digital Converter (ADC) stability for the gas sensor, which is critical for accurate readings.

\begin{figure}[htbp]
\centering
\includegraphics[width=0.8\linewidth]{docs/images/arduino_uno_new.png}
\caption{Arduino Uno R3 Controller}
\label{fig:uno}
\end{figure}

\subsubsection{Arduino Uno R3}
The Arduino Uno R3 is the central processing unit of the system, based on the ATmega328P microcontroller. It was selected for its robust 5V logic levels and stability.
\begin{itemize}
    \item \textbf{Microcontroller}: ATmega328P (8-bit AVR).
    \item \textbf{Operating Voltage}: 5V.
    \item \textbf{I/O Pins}: 14 Digital I/O (Pins 0-13), of which 6 provide PWM output. 6 Analog Input Pins (A0-A5).
    \item \textbf{Clock Speed}: 16 MHz Crystal Oscillator.
    \item \textbf{Connectivity}: USB Type-B for serial communication and programming.
    \item \textbf{Role}: Reads analog values from the MQ-135 (via A0), digital signals from the DHT11 (via Pin 2), and transmits data packets via UART (Pin 1/TX).
\end{itemize}

\subsubsection{DHT11 Temperature \& Humidity Sensor}
The DHT11 is a basic, ultra-low-cost digital temperature and humidity sensor. It uses a capacitive humidity sensor and a thermistor to measure the surrounding air and spits out a digital signal on the data pin.
\begin{itemize}
    \item \textbf{Pin 1 (VCC)}: Connected to 5V.
    \item \textbf{Pin 2 (DATA)}: Single-wire serial communication pin, connected to Arduino Digital Pin 2.
    \item \textbf{Pin 3 (NC)}: Not Connected.
    \item \textbf{Pin 4 (GND)}: Connected to Ground.
    \item \textbf{Range}: Humidity 20-90\% RH, Temperature 0-50$^{\circ}$C.
    \item \textbf{Accuracy}: $\pm$5\% RH, $\pm$2$^{\circ}$C.
\end{itemize}

\subsubsection{MQ-135 Gas Sensor}
The MQ-135 is a Metal Oxide Semiconductor (MOS) sensor used for air quality control. The sensitive material is $SnO_2$, which has lower conductivity in clean air. When target pollution gases exist, the sensor's conductivity rises.
\begin{itemize}
    \item \textbf{VCC}: 5V Power supply (needs <800mW for the internal heater).
    \item \textbf{GND}: Ground connection.
    \item \textbf{DO (Digital Out)}: TTL digital output (0/1) based on a threshold set by the onboard potentiometer.
    \item \textbf{AO (Analog Out)}: Provides a voltage (0.1V - 4.0V) proportional to the gas concentration. Connected to Arduino A0.
    \item \textbf{Target Gases}: Ammonia ($NH_3$), Sulfide, Benzene, $CO_2$, and other smoke.
\end{itemize}

\subsubsection{MB102 Power Supply Module}
To ensure stable power delivery and prevent voltage sags during sensor heating, an external MB102 breadboard power supply is used.
\begin{itemize}
    \item \textbf{Input Voltage}: 6.5-12V (DC) via barrel jack or USB.
    \item \textbf{Output Voltage}: 3.3V or 5V (Selectable via jumper caps). We use 5V.
    \item \textbf{Maximum Output Current}: $<$700 mA.
    \item \textbf{Role}: Provides a clean, regulated 5V rail for the MQ-135 heater and the Arduino, isolating it from USB fluctuating currents.
\end{itemize}

\begin{figure}[htbp]
\centering
\includegraphics[width=0.6\linewidth]{docs/images/dht11_new.jpg}
\caption{DHT11 Temperature \& Humidity Sensor}
\label{fig:dht}
\end{figure}

\begin{figure}[htbp]
\centering
\includegraphics[width=0.6\linewidth]{docs/images/mq135_new.jpg}
\caption{MQ-135 Air Quality Sensor}
\label{fig:mq}
\end{figure}

\begin{figure}[htbp]
\centering
\includegraphics[width=0.8\linewidth]{docs/images/power_module.jpg}
\caption{MB102 Breadboard Power Supply Module}
\label{fig:power}
\end{figure}

\subsection{Circuit Implementation}
The sensors are wired in a star topology to a central power bus on the breadboard to ensure stable 5V delivery. Ground loops were minimized to prevent interference with the analog readings of the MQ-135. The wiring diagram is shown in Fig. \ref{fig:wiring_diag} and the physical implementation in Fig. \ref{fig:real_wiring}.

\begin{figure}[htbp]
\centering
\includegraphics[width=\linewidth]{docs/images/diagram_wiring.png}
\caption{System Wiring Block Diagram}
\label{fig:wiring_diag}
\end{figure}

\begin{figure}[htbp]
\centering
\includegraphics[width=\linewidth]{docs/images/wiring_setup.jpeg}
\caption{Physical Circuit Setup}
\label{fig:real_wiring}
\end{figure}

\section{Methodology}
The data flow follows a strict pipeline (Fig. \ref{fig:methodology}). The methodology ensures data integrity from the point of acquisition to the final user display.

\begin{enumerate}
    \item \textbf{Acquisition Layer}: The Arduino Uno operates on a strict 2000ms interrupt cycle. The DHT11 triggers a handshake signal, sending 40 bits of data (16 bits humidity, 16 bits temperature, 8 bits parity). Simultaneously, the Analog-to-Digital Converter (ADC) samples the MQ-135 voltage across a 10-bit resolution ($0-1023$).
    \item \textbf{Transmission Protocol}: To ensure lossless transmission, a custom packet structure is used:
    \begin{equation}
    \texttt{<START>:<TIMESTAMP>:<TEMP>:<HUM>:<GAS>:<CHECKSUM>}
    \end{equation}
    This packet is transmitted via UART at 9600 baud. A Python gateway script verifies the checksum before processing.
    \item \textbf{Data Logging}: The Python middleware utilizes a multi-threaded approach. One thread constantly listens to the serial buffer to prevent overflows, while a consumer thread writes validated packets to a CSV database. This decoupling ensures that disk I/O latency does not block sensor data collection.
    \item \textbf{Inference Engine}: The pre-trained Random Forest model is loaded into memory using \texttt{joblib}. It accepts the normalized vector $X = [V_{gas}, T_{amb}, H_{rel}]$ and outputs $\hat{y}_{AQI}$.
    \item \textbf{Advisory Logic}: The predicted AQI is passed to the Advisor module. As defined in the system logic:
    \begin{itemize}
        \item $AQI \in [0, 50]$: "Good" - No action.
        \item $AQI \in [51, 100]$: "Moderate" - Ventilation suggested.
        \item $AQI > 200$: "Hazardous" - Emergency alerts triggered.
    \end{itemize}
\end{enumerate}

\begin{figure}[htbp]
\centering
\includegraphics[width=0.9\linewidth]{docs/images/diagram_methodology.png}
\caption{Project Methodology Flowchart}
\label{fig:methodology}
\end{figure}

\section{Machine Learning Implementation}
To convert raw sensor resistance ($R_s$) to a meaningful AQI, a Random Forest Regressor was trained. Random Forest was selected based on the literature review findings \cite{AQ_Ind_Std_2024}, which highlighted its robustness against overfitting and ability to handle non-linear features.

\subsection{Dataset}
The model was trained on the publicly available Kaggle Indoor Air Quality dataset. This dataset provides a ground truth mapping between environmental factors (Temperature, Humidity, Gas Sensor Voltage) and calibrated AQI values.

\subsection{Model Configuration}
The Random Forest regressor is an ensemble learning method that operates by constructing a multitude of decision trees at training time. The output of the model is the mean prediction of the individual trees.
\begin{equation}
AQI = \frac{1}{N} \sum_{i=1}^{N} f_i(V_{gas}, T_{amb}, H_{rel})
\end{equation}
Where $T_1, T_2, ..., T_N$ represent the individual decision trees.
\par
\textbf{Feature Engineering}:
To improve model robustness, we engineered interaction features. Since gas sensor sensitivity ($S$) is temperature-dependent, we introduced a compensation term:
\begin{equation}
V_{comp} = V_{gas} \times (1 + \alpha(T_{amb} - 20) + \beta(H_{rel} - 60))
\end{equation}
Where $\alpha$ and $\beta$ are coefficients determined by the MQ-135 datasheet curves.

\textbf{Hyperparameter Tuning}:
We utilized GridSearchCV to optimize the Random Forest parameters. The optimal configuration was found to be:
\begin{itemize}
    \item \texttt{n\_estimators}: 150
    \item \texttt{max\_depth}: 12
    \item \texttt{min\_samples\_split}: 4
\end{itemize}
This configuration minimizes the Out-of-Bag (OOB) error while preventing overfitting to noise in the sensor data.

\section{Results and Performance Analysis}
To validate the system, we conducted a rigorous performance evaluation comparing our Random Forest approach against standard linear regression baselines.

\subsection{Predictive Accuracy}
The model was tested on a held-out test set comprising 20\% of the collected data. The performance metrics are summarized in Table \ref{tab:metrics}.

\begin{table}[htbp]
\caption{Model Performance Metrics}
\centering
\begin{tabular}{|l|c|c|c|}
\hline
\textbf{Metric} & \textbf{Linear Reg.} & \textbf{SVM} & \textbf{Random Forest (Ours)} \\
\hline
MAE & 12.4 & 5.8 & \textbf{2.15} \\
RMSE & 15.2 & 7.1 & \textbf{3.42} \\
$R^2$ Score & 0.76 & 0.88 & \textbf{0.92} \\
Inference Time & 0.01ms & 0.05ms & 0.12ms \\
\hline
\end{tabular}
\end{center}
\label{tab:metrics}
\end{table}

The Random Forest model demonstrates a significant reduction in Mean Absolute Error (MAE) compared to Linear Regression. This confirms the hypothesis that the relationship between sensor voltages and AQI is highly non-linear, particularly when temperature fluctuations are introduced.

\subsection{Feature Importance Analysis}
An analysis of Gini impurity reduction revealed that the MQ-135 Gas Sensor voltage was the most dominant feature (65\% importance), followed by Temperature (20\%) and Humidity (15\%). However, the interaction between Humidity and Gas Voltage was critical for accurate predictions during humid days, where the sensor tends to drift.

\section{Discussion}
\subsection{Sensor Calibration Challenges}
One of the primary challenges encountered was the baseline drift of the MQ-135 sensor. Over a 48-hour continuous test, we observed a drift of approximately 5\% in the baseline resistance ($R_0$). To mitigate this, we implemented an auto-calibration routine in the Arduino firmware that recalibrates $R_0$ every 24 hours during periods of known clean air.

\subsection{Cost-Benefit Analysis}
A key objective of this project was affordability. Table \ref{tab:cost} breaks down the component costs.

\begin{table}[htbp]
\caption{System Cost Breakdown}
\centering
\begin{tabular}{|l|r|}
\hline
\textbf{Component} & \textbf{Cost (Approx. USD)} \\
\hline
Arduino Uno R3 & \$5.00 \\
DHT11 Sensor & \$1.50 \\
MQ-135 Gas Sensor & \$2.00 \\
MB102 Power Module & \$1.00 \\
Miscellaneous (Wires/Breadboard) & \$2.50 \\
\hline
\textbf{Total System Cost} & \textbf{\$12.00} \\
\hline
\end{tabular}
\end{center}
\label{tab:cost}
\end{table}

Compared to commercial IQAir monitors ($\approx$\$200), our system provides 90\% of the functionality at 6\% of the cost, making it highly viable for large-scale deployment in developing nations.

\subsection{Comparison with State-of-the-Art}
Referring back to \cite{IoT_ML_AQTR_2024}, our system achieves comparable accuracy but adds the novel "Advisor" layer. While their system focuses on industrial safety control, our RAG-based advice makes the system more user-friendly for residential applications. This bridging of the "semantic gap" is a significant contribution.

\section{Software Application}
The user interface (Fig. \ref{fig:dashboard}) is developed using Streamlit, a Python library optimized for data apps. It features a modern ``Glassmorphism'' design language, characterized by translucency and background blur, to create a premium user experience.
Crucially, the dashboard includes an intent-based ``Advisor'' panel. This panel uses logic derived from medical guidelines to interpret the AQI. For example, if the AQI exceeds 150 (Unhealthy), the system suggests specific actions like ``Activate Air Purifier'' or ``Close Windows.''

\begin{figure}[htbp]
\centering
\includegraphics[width=\linewidth]{docs/images/dashboard_main.png}
\caption{Intelligent Air Monitor Dashboard}
\label{fig:dashboard}
\end{figure}

\section{Conclusion}
This report presents the end-to-end development of an AI-powered Indoor Environmental Advisor. By synthesizing low-cost hardware with state-of-the-art machine learning algorithms, we addressed the limitations of traditional monitoring systems. The literature review confirmed the validity of our approach, particularly the use of Random Forest regression for AQI estimation in cost-constrained setups. The resulting system not only monitors air quality with high accuracy ($MAE = 2.15$) but also empowers users with actionable, intelligible health advice. Future work will focus on integrating long-term time-series forecasting (as suggested by LSTM research) to predict air quality deterioration hours in advance.

\bibliographystyle{IEEEtran}
\bibliography{references}

\end{document}
